% !TeX spellcheck = ru_RU
% !TEX root = vkr.tex

\section*{Заключение}
% \textbf{Обязательно для промежуточного, полугодового, годового и  любых других отчётов.} Кратко, что было сделано.

% \textbf{Для практик/ВКР.} Также важно сделать список результатов, который будет один к одному соответствовать задачам из раздела~\ref{sec:task}.

% \begin{itemize}
% \item Результат к задаче №1.
% \item Результат к задаче №2.
% \item и т.д.
% \end{itemize}
% \noindent Для промежуточных отчетов сюда важно записать какие задачи уже были сделаны за осенний семестр, а какие только планируется сделать.

% Также сюда можно написать планы развития работы в будущем, или, если их много, выделить под это отдельную предпоследнюю главу.

В результате проведенной работы были выполнены следующие задачи:
\begin{enumerate}
	\item написана реализация парсера декларативного языка Myfile;
	\item проведено тестирование парсера на существующей кодовой базе в репозитории Embox.
\end{enumerate}

Исходный код работы доступен по ссылке \footnote{\href{https://github.com/mikhaylovilya/caravan}{https://github.com/mikhaylovilya/caravan} (дата обращения: 08.01.2024)}.
Имя аккаунта: mikhaylovilya. Название репозитория: caravan.
