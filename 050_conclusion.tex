% !TeX spellcheck = ru_RU
% !TEX root = vkr.tex

\section*{Заключение}
% Список результатов, который будет либо один к одному соответствовать задачам из раздела~\ref{sec:task}, либо их уточнять (например, если было \enquote{выбрать}, то тут \enquote{выбрано то-то}).
В результате работы были выполнены следующие задачи:
\begin{itemize}
    \item проведен обзор динамических бинарных трансляторов и сравнение их с Instrew;
    \item проведен обзор архитектуры Instrew;
    \item реализована минимальная функциональность, необходимая для запуска Instrew:
          \begin{itemize}
              \item выполнены процессорно-специфичные патчи на языке ассемблера;
              \item реализована эмуляция системных вызовов;
              \item реализовано четыре архитектурно-специфичных релокаций.
          \end{itemize}
\end{itemize}

Протестировать Instrew на простых примерах не удалось, из-за сложности локализации ошибки в реализации релокаций.
Код доступен по ссылке: \href{https://github.com/mikhaylovilya/instrew/tree/host-rv64-stash}{https://github.com/mikhaylovilya/instrew/tree/host-rv64-stash}.
% \noindent Если работа на несколько семестров, отчитывайтесь только за текущий.
% Можно в свободной форме обрисовать планы продолжения работы, но не увлекайтесь~--- если работа будет продолжена, по ней будет ещё один отчёт.

% В заключении \emph{обязательна} ссылка на исходный код, если он выносится на защиту, либо явно напишите тут, что код закрыт.
% Если работа чисто теоретическая и это понятно из решённых задач, про код можно не писать.
% Обратите внимание, что ссылка на код должна быть именно в заключении, а не посреди раздела с реализацией, где её никто не найдёт.

% Старайтесь оформить программные результаты работы так, чтобы это был один репозиторий или один пуллреквест, правильно оформленный~--- комиссии тяжело будет собирать Ваши коммиты по всей истории.
% А если над проектом работало несколько человек и всё успело изрядно перемешаться, неизбежны вопросы о Вашем вкладе.

% Заключение люди реально читают (ещё до \enquote{основных} разделов работы, чтобы понять, что же получилось и стоит ли вообще работу читать), так что оно должно быть вылизано до блеска.
