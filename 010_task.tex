% !TeX spellcheck = ru_RU
% !TEX root = vkr.tex

% \section{Постановка задачи}
% \label{sec:task}
\section{Введение}
Бинарные трансляторы являются инструментами для эмуляции, бинарного анализа и профиляции, важным классом которых являются динамические бинарные трансляторы, которые преобразуют двоичный ход на лету и генерируют нужные инструкции по мере требования. Примером такого транслятора является Rosetta 2. Часто, исходный двоичный код преобразуется в какое-либо промежуточное представление, где такое преобразование осуществляет лифтер. Примерами такого представления являются TCG для QEMU. Но в качестве промежуточного представления хочется использовать LLVM IR, так как тулчейн LLVM может делать качественные оптимизации кода. В данной работе хотим рассмотреть динамические бинарные трансляторы с поднятием двоичного кода в LLVM IR для RISC-V.
\section{Обзор}
Для начала необходимо рассмотреть такие канонические ДБТ, как
\begin{itemize}
    \item QEMU-user, который является стандартом эмуляции userland программ;
    \item Rosetta 2, который является эмулятором с закрытым исходным кодом от Apple;
    \item Valgrind, который является инструментом динамического бинарного анализа в состав которого входит Memcheck и Callgrind.
\end{itemize}

Также довольно популярным является Box64, который позволяет запускать программы, в частности игры, скомпилированные под linux x86/x64, на ARM64 и RISC-V, и который отличается производительностью.
Из ДБТ с поднятием в LLVM IR, можно выделить
\begin{itemize}
    \item Instrew в связке с лифтером Rellume;
    \item BinRec (не обновлялся с 2022го года);
    \item HQEMU (не обновлялся с 2018го года);
    \item DBILL, LnQ, Lasagne (есть только статьи, либо код закрыт).
\end{itemize}

В результате обзора стало понятно, что, судя по бенчмаркам, Instrew + Rellume производительнее, чем QEMU и Valgrind, благодаря оптимизациям LLVM. Также, в сравнение: с QEMU, Valgrind и Box64, является не только эмулятором, но и инструментом для динамического бинарного анализа (что позволяет менять семантику транслируемых программ), так как из коробки имеет API для добавления анализаторов, например, подсчет инструкций. По этим причинам, неплохо было бы портировать Instrew на RISC-V
Также в сравнение с другими ДБТ с поднятием в LLVM IR поддерживается контрибьютором и не заброшен, а также имеет поддержку архитектуры RISC-V, как целевой.

\section{Постановка задачи}
Целью работы является добавление поддержки host-архитектуры RISC-V. Для ее выполнения были поставлены следующие задачи:
\begin{itemize}
    \item выполнить обзор архитектуры Instrew;
    \item реализовать минимальную функциональность для запуска Instrew на RISC-V, а именно:
          \begin{itemize}
              \item скомпилировать, добавив процессорно-специфические патчи и проставив константы (макросы riscv, размеры функций, адреса и номера и аргументы системных вызовов);
              \item реализовать функции, связанные с RISC-V ABI (эмуляция системных вызовов, трамплины PLT таблицы, релокации).
          \end{itemize}
    \item протестировать на простых примерах.
\end{itemize}

\section{Архитектура Instrew}
Instrew реализует архитектуру клиент-сервер. Клиент написан на языке Си и из его основных функций можно выделить:
\begin{itemize}
    \item отправка функций транслируемой программы серверу;
    \item получение от сервера ELF объекта, проведение релокаций, разрешение символов и эмуляция системных вызовов.
\end{itemize}
Таким образом клиент выполняет диспетчеризацию исходных инструкций и является динамическим линкером. Также из деталей реализации важно отметить, что для клиента написана своя реализация libc (так называемый minilibc).

Сервер написан на C++ с помощью LLVM, и его основными функциями являются:
\begin{itemize}
    \item поднятие двоичного кода в LLVM IR с помощью Rellume;
    \item применение оптимизаций и кодогенерация.
\end{itemize}

В данной практике была проведена работа именно над клиентской частью Instrew.

\section{Реализация (1/2)}
Первый шаг реализации --- это скомпилировать Instrew. Для этого необходимо проставить константы, а именно макросы RISC-V, размеры функций, адреса, номера и аргументы системных вызовов. Далее необходимо написать на ассемблере точку входа в программу в нашем minilibc (в glibc --- это файл start.S, в musl --- crt\_arch.S). Из сложностей можно отметить, что при написании точки входа был просмотрен код канонических реализаций libc и там была обнаружена плохо задокументированная псевдоинструкция, которая, грубо говоря, не понятно, что делала, и при подставлении которой в свой код, все магически работало.

\section{Особенности Instrew (1/2)}
% После того как мы успешно написали точку входа в программу, необходимо разобраться в ошибке сегментации, наступающей после запуска. После локализации ошибки оказалось, что мы не можем выполнить релокации своих функций. После диагностирования с помощью pmap, оказалось, что к клиенту динамически линкуется ld-linux, чего быть не должно. Проблема была в флаге --static-pie, который в версии gcc-13 
После того, как мы успешно написали точку входа в программу, необходимо попробовать запустить клиент Instrew без аргументов. И здесь хочется рассказать, почему Instrew сложно дебажить:
\begin{itemize}
    \item Во-первых, из-за специфических флагов сборки. Пытаемся запустить программу --- получаем ошибку сегментации. Asan и gprof здесь не работают:
    \begin{itemize} 
	\item первый --- по причине того, что система сборки автоматически его выключает;
	\item второй --- потому что при попытке поставить флаг -pg, система сборки выдает ошибку --- конфиликтующие флаги.
    \end{itemize}
    После локализации проблемы, оказалось, что динамически подключался линкер ld-linux, а такого быть не должено. Оказалось проблема в том, что --static-pie флаг в gcc одной и той же версии 13, интерпретируется по-разному на архитектурах x86\_64 и RISC-V.
    \item Во-вторых, из-за атрибутов и релокаций. Так как клиент Instrew самостоятельно разрешает свои релокации, было неочевидно, почему программа выходит с ошибкой. После проверки правильности подсчета всех релокаций, оказалось, что есть функция с атрибутом, для которой тип релокации отличается от всех остальных функций (или просто не матчится). То есть атрибут меняет тип релокации, что было неочевидно.
    \item В-третьих, сервер запускает клиент с помощью fork() и fexecve(). Поэтому когда пытаемся запустить клиент Instrew в gdb, ничего не выходит. Точку остановки поставить нельзя, символы не подгружаются. Спустя некоторое время научились запускать Instrew сервер с правильными дебаг флагами, чтобы можно было запустить gdb на клиенте.
    \item Вывод:
	\begin{itemize}
	    \item из-за некоторых флагов сборки и линковки, а также особенностей реализации сложно дебажить Instrew;
	    \item по сути из инструментов есть только printf, gdb и Valgrind.
	\end{itemize}

\end{itemize}

% Дословно \enquote{Целью работы является... Для её выполнения были постав\-лены следующие задачи:}
% \begin{enumerate}
%     \item реализовать это (раздел~\ref{subsec:task1});
%     \item спроектировать то-то (раздел~\ref{subsec:task2}) наилучшим образом;
%     \item протестировать на том-то (раздел~\ref{subsec:task3}) и обогнать тех-то;
%     \item \sout{изучить язык \OCaml{}} писать тут не надо, так как тут должны быть задачи, выполнение которых можно проверить/оценить прочитав текст или выслушав доклад;
%           (т.е. Ваши достижения должны быть опровержимы)
%           \begin{itemize}
%               \item это может вызвать сомнения по поводу обзора~--- \emph{выполнить обзор} писать можно и нужно, но защищаемым результатом будут не ваши знания, а текст обзора (то есть он должен иметь ценность сам по себе);
%           \end{itemize}
%     \item обязательна задача на валидацию результата, будь то эксперимент, апробация, внедрение~--- то есть доказательство того, что Вы сделали что-то, нужное пользователю.
%           Не путайте с валидацией~--- доказательством того, что Вы сделали то, что хотели Вы (например, тесты~--- валидация результата, хорошо, но недостаточно).
% \end{enumerate}
