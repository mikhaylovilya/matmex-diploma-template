% !TeX spellcheck = ru_RU
% !TEX root = vkr.tex

\section*{Введение}
\thispagestyle{withCompileDate}

% Бинарная трансляция -- форма трансляции, где последовательность инструкции одной архитектуры процессора транслируется в другую архитектуру, либо ту же. Это осущестляется для ряда различных задач, например: эмуляция программ guest-архитектуры/платформы на host-архитектуре/платформе, оптимизация, профилирование и т.д. Классифицировать бинарные трансляторы можно по принципу действия: статические и динамические; по цели: эмуляция, оптимизация, профилирование. Примерами популярыных open-source dynamic binary instrumentation frameworks являются QEMU\footnote{cite qemu 2005 paper}, Rosetta, Valgrind\footnote{cite valgrind paper}.

%\footnote{\href{https://www.usenix.org/legacy/event/usenix05/tech/freenix/full\_papers/bellard/bellard.pdf}{https://www.usenix.org/legacy/event/usenix05/tech/freenix/full\_papers/bellard/bellard.pdf}}

Бинарная трансляция --- форма трансляции, где на вход подается последовательность инструкций одной архитектуры процессора, а на выходе получаем инструкции другой архитектуры. Такая трансляция может осуществляться с целью эмулировать какой-либо набор инструкций, чтобы запускать программы, скомпилированные не под назначенную архитектуру, либо же скомпилированные под другое расширение набора команд (важный пример --- расширения RISC-V ISA).  Примерами эмуляторов являются: QEMU\footnote{\href{https://www.qemu.org/}{https://www.qemu.org/}}, Rosetta 2, Box64\footnote{\href{https://box86.org/}{https://box86.org/}}. Также исходные инструкции транслировать в инструкции той же архитектуры с помощью инструментов бинарного анализа для выполнения оптимизаций и внесения патчей, например, когда рекомпиляция/декомпиляция осложнена; для профиляции и отладки транслируемой программы. Примерами таких инструментов являются Valgrind\footnote{\href{https://valgrind.org/}{https://valgrind.org/}}, Pin, Dynamo.

Бинарную трансляцию можно классифицировать по способу выполнения на статическую и динамическую.
%Все вышеприведенные инструменты являются динамическими бинарными трансляторами (далее --- ДБТ)
Также бинарные трансляторы могут использовать поднятие инструкций до некоторого более высокоуровневого промежуточного представления. QEMU использует TCG, Valgrind использует VEX. Но хочется использовать более распространенное промежуточное представление, позволяющее делать качественные оптимизации. Одним из таких является LLVM IR.

Хочется иметь динамический бинарный транслятор с поднятием в LLVM IR у которого есть поддержка архитектуры RISC-V, как хоста.

Таким критериям подходит динамический бинарный транслятор (далее --- ДБТ) Instrew с поднятием инструкций в LLVM IR, но он не запускается на процессорах с архитектурой RISC-V. Поэтому предлагается реализовать поддержку RISC-V для Instrew.

% Формат из 4х частей рекомендуется в курсе Д.~Кознова~\cite{koznov} по написанию текстов.

% \begin{enumerate}
%     \item Известная информация (background/обзор).
%     \item Неизвестная информация (пробел в знаниях, \enquote{Gap}).
%     \item Гипотезы, вопросы, цели~--- \enquote{что болит}, что будет решать Ваша работа.
%     \item Подход, план решения задачи, предлагаемое решение.
% \end{enumerate}

% Последний абзац должен читаться и быть понятен в отрыве от других трёх.
% Никакие абзацы нумеровать нельзя.

% Части (абзацы) должны занять максимум две страницы, идеально уложиться в одну.

% С.-П. Джонс~\cite{SPJGreatPaper} предлагает несколько другой формат написания введения.
% Вполне возможно, что если Ваша работа про языки программирования, то его формат будет удачнее.

% Введение и заключение читают чаще всего, поэтому они должны быть \enquote{вылизаны} до блеска.

% \blfootnote{
%     Иногда рецензенту полезно знать какого числа компилировался текст, чтобы оценить актуальность версии текста.
%     В этом случае полезно вставлять в текст дату сборки.
%     Для совсем официальных релизов документа это не вполне канон.\\
%     Также здесь имеет смысл указать, если работа сделана на деньги, например, Российского Фонда Фундаментальных Исследований (РФФИ) по гранту номер такой-то, и т.п.}
