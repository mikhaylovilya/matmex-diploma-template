% !TeX spellcheck = ru_RU
% !TEX root = vkr.tex

\section*{Введение}
\thispagestyle{withCompileDate}

\paragraph{} Системы сборки --- это неотъемлемая часть разработки программного обеспечения, так как позволяет автоматизировать некоторые однотипные задачи, освобождая ресурсы на что-то другое. Системы сборки также применяются для создания конкретных образов конфигурируемых ОС, которые включают необходимые для какой-либо задачи модули. Представителем такого класса ОС является ОСРВ (операционная система реального времени) Embox, которая использует систему сборки Mybuild.

% Системы сборки являются довольно сложным программным продуктом. Mybuild написана на языке Makefile и состоит из следующих  библиотек функций - расширения языка Makefile
% Можно привести несколько примеров систем сборок: Makefile, CMake, Ninja, Dune, Meson. Основная задача системы сборки - это
Можно сформулировать задачу системы сборки так: собрать актуальную версию проекта на основе зависимостей между его частями и данных об их изменении, выполнив минимум задач по перекомпилированию \cite{mokhov2018build}. Главный пример системы сборки - это GNU Make, написанный на языке C и повсеместно использующийся на Unix-подобных ОС. Результатом работы некоторых систем сборки являются make-файлы, что позволяет повысить уровень абстракции при описании зависимостей. Примером таких систем являются CMake, Meson и Mybuild.

Реализация Mybuild написана на языке GNU Make, из-за чего возникают проблемы с масштабированием данной системы. Также код плохо задокументирован, в следствие чего работать с данным проектом становится затруднительно.

Предлагается провести реинжиниринг системы сборки Mybuild, а именно написать реализацию на функциональном языке OCaml.
% Формат из 4х частей рекомендуется в курсе Д.~Кознова~\cite{koznov} по написанию текстов.
% Части (абзацы) должны занять максимум на две страницы.
% \begin{enumerate}
% 	\item Известная информация (background/обзор).
% 	\item Неизвестная информация (пробел в знаниях).
% 	\item Гипотезы, вопросы, цели.
% 	\item Подход, план решения задачи, предлагаемое решение.
% \end{enumerate}

% Последний абзац должен читаться и быть понятен в отрыве от других трёх. Никакие абзацы нумеровать нельзя.

% С.-П. Джонс~\cite{SPJGreatPaper} предлагает несколько другой формат написания введения.
% Вполне возможно, что если Ваша работа про языки программирования, то его формат будет удачнее.

% \blfootnote{
% 	Иногда рецензенту полезно знать какого числа компилировался текст, чтобы оценить актуальность версии текста. В этом случае полезно вставлять в текст дату сборки. Для совсем официальных релизов документа это не вполне канон.\\
% 	Также здесь имеет смысл указать, если работа сделана на деньги, например, Российского Фонда Фундаментальных Исследований (РФФИ) по гранту номер такой-то, и т.п.}

% Embox - это кофигурируемая операционная система реального времени, которая применяется для embedded-систем.
% Mybuild - это система сборки для конфигурируемой операционной системы Embox, реализация которой написана на Makefile.
% Она состоит из двух декларативных языков программирования: Configfile, который отвечает за описание модулей, которые необходимо включить в сборку операционной системы, и Myfile, который ...
% Зачем нужна своя система сборки, когда есть CMake например???
